\section{Related Work}

Whenever we are doing problems like Sudoku, there are always many ways to approach them. Starting with the backtracking algorithm, this is one of the best algorithms to solve such problems. We have made solving Sudoku problem fantastically simple using this algorithm.
The idea behind backtracking algorithm is very easy to understand. Any time we have a problem that can be solved by a series of decisions, we might make a wrong decision but when we realise that we have made a wrong decision, we can backtrack to the place where we have made a decision and we can try something else.
What was tricky about backtracking is not the concept but figuring out how to implement it in the code and apply it in program. Backtracking is most often associated with recursion but the way backtracking happens in a recursive solution is tough to see and that can make it difficult at first to write recursive backtracking solutions.

Search algorithms can be used to tackle artificial intelligence problems. Uninformed search and informed search with heuristics are two types of search algorithms. Uninformed search is a search method that is able to only differentiate between a target state and a non-goal state and has no knowledge about how distant the goal state might be from the present state \cite{bib_search}. In other words, uninformed search techniques are strategies which use a tree-traversing algorithm to search the sequence through which an agent could reach to the goal state without using any additional information. Depth-first search (DFS) and Breadth-First search (BFS) are among the uniformed search strategies.

As it is apparent, there is a set of constraints that are to be satisfied in every given Sudoku puzzle. Hence, Sudoku is recognized as a classic example of constraint satisfaction problem. Helmut in his paper \cite{bib_constraint} found a method using quasi-group completion concepts by which the Sudoku solution is obtained without performing any search algorithms. This can significantly reduce the time and memory usage and indicates that constraints satisfaction can be an efficient choice for solving Sudoku puzzles.

Genetic algorithm is a search heuristic method to solve the problem, self-discovery and make judgement quickly and efficiently. this algorithm is inspired by Darwinism, which is a biological evolution theory that all species arise through natural selection. A genetic algorithm generates a patch of generations with possible solutions for the research problem and selects the fittest individuals which are parents to reproduce the next child generation. Evaluate them to get the fittest children as the next parents to get the finalized best solution.
